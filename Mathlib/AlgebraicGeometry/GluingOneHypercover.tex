\documentclass{article}
\usepackage[utf8]{inputenc}
\usepackage{amsmath, amsthm, amssymb}
\usepackage{xcolor}
\usepackage{listings}
\usepackage{hyperref}
\usepackage{geometry}
\geometry{margin=1in}

% Define colors for Lean syntax highlighting
\definecolor{leanKeyword}{RGB}{0, 0, 255}       % Blue for keywords
\definecolor{leanType}{RGB}{139, 0, 139}        % Dark magenta for types
\definecolor{leanString}{RGB}{0, 128, 0}        % Green for strings
\definecolor{leanComment}{RGB}{128, 128, 128}   % Gray for comments
\definecolor{leanDefinition}{RGB}{255, 140, 0}  % Dark orange for definitions
\definecolor{leanAttribute}{RGB}{255, 20, 147}  % Deep pink for attributes
\definecolor{leanBackground}{RGB}{250, 250, 250} % Light gray background

% Define Lean language for listings
\lstdefinelanguage{Lean}{
  % Keywords
  morekeywords=[1]{def, theorem, lemma, instance, class, structure, inductive,
    axiom, constant, variable, namespace, section, noncomputable, protected,
    private, abbrev, example, open, attribute, local, mutual, by, where,
    extends, deriving, end, import, prelude, theory, with, without, using,
    match, if, then, else, fun, have, show, from, let, in, do, begin, calc},
  % Types and Type constructors
  morekeywords=[2]{Type, Prop, Sort, Nat, Int, Bool, String, Char, List,
    Option, Unit, Prod, Sum, Sigma, Subtype, Set, Scheme, CommRingCat,
    AffineScheme, Opens, IsAffine, IsAffineOpen, Spec, Category, Functor,
    Iso, IsIso, CompactSpace, IsCompact, IsOpenImmersion, Equiv, Hom,
    pullback, PullbackCone, OpenCover, GlueData, Multicoequalizer,
    OneHypercover, Sheaf, zariskiTopology, Multifork, PUnit},
  % Tactics (if they appear)
  morekeywords=[3]{simp, rfl, exact, apply, intro, cases, induction, refl,
    constructor, assumption, contradiction, trivial, sorry, admit, rw,
    convert, infer_instance, ext, rintro, obtain, refine, dsimp},
  % Sensitive
  sensitive=true,
  % Comments
  morecomment=[l]{--},
  morecomment=[n]{/-}{-/},
  % Strings
  morestring=[b]",
  % Style
  keywordstyle=[1]\color{leanKeyword}\bfseries,
  keywordstyle=[2]\color{leanType}\bfseries,
  keywordstyle=[3]\color{leanDefinition}\itshape,
  commentstyle=\color{leanComment}\itshape,
  stringstyle=\color{leanString},
  % Highlight attributes
  moredelim=[is][\color{leanAttribute}]{@[}{]},
  moredelim=[is][\color{leanAttribute}]{\#[}{]},
}

\lstset{
  language=Lean,
  basicstyle=\ttfamily\small,
  backgroundcolor=\color{leanBackground},
  breaklines=true,
  frame=single,
  framerule=0.5pt,
  rulecolor=\color{gray!30},
  numbers=left,
  numberstyle=\tiny\color{gray},
  numbersep=5pt,
  showstringspaces=false,
  captionpos=b,
  xleftmargin=10pt,
  framexleftmargin=10pt,
  % Unicode replacements for mathematical symbols
  literate={Γ}{{$\Gamma$}}1
           {⊤}{{$\top$}}1
           {⟶}{{$\to$}}2
           {≅}{{$\cong$}}1
           {⊆}{{$\subseteq$}}1
           {⨆}{{$\bigsqcup$}}1
           {∈}{{$\in$}}1
           {∃}{{$\exists$}}1
           {∧}{{$\land$}}1
           {⁻¹ᵁ}{{$^{-1}U$}}3
           {⁻¹}{{$^{-1}$}}2
           {≫}{{>>}}1
           {≤}{{$\leq$}}1
           {ᵒᵖ}{{$^{op}$}}2
           {⥤}{{$\Rightarrow$}}2
           {≌}{{$\simeq$}}1
           {⋙}{{$\ggg$}}1
           {∀}{{$\forall$}}1
           {↑}{{$\uparrow$}}1
           {𝒰}{{$\mathcal{U}$}}1
           {𝟙}{{$\mathbb{1}$}}1
           {''ᵁ}{{''U}}2
           {≪≫}{{$\ll\gg$}}2
           {ι}{{$\iota$}}1
           {→}{{$\to$}}1
           {ᵁ}{{U}}1
           {:=}{{$\mathrel{:=}$}}2
           {×}{{$\times$}}1
           {⧸}{{$/\!/$}}1
           {↔}{{$\leftrightarrow$}}1
           {⟨}{{$\langle$}}1
           {⟩}{{$\rangle$}}1
           {α}{{$\alpha$}}1
           {β}{{$\beta$}}1
           {γ}{{$\gamma$}}1
           {δ}{{$\delta$}}1
           {ε}{{$\varepsilon$}}1
           {λ}{{$\lambda$}}1
           {μ}{{$\mu$}}1
           {σ}{{$\sigma$}}1
           {τ}{{$\tau$}}1
           {φ}{{$\varphi$}}1
           {ψ}{{$\psi$}}1
           {ω}{{$\omega$}}1
           {∞}{{$\infty$}}1
           {∂}{{$\partial$}}1
           {∇}{{$\nabla$}}1
}

\theoremstyle{definition}
\newtheorem{definition}{Definition}
\newtheorem{theorem}{Theorem}
\newtheorem{lemma}{Lemma}
\newtheorem{instance}{Instance}

\title{One-Hypercover from Glue Data in Mathlib4\\
\large A Companion to \texttt{GluingOneHypercover.lean}}
\author{}
\date{}

\begin{document}
\maketitle

\section{Introduction}

This document provides a natural language companion to the \texttt{GluingOneHypercover.lean} file in Mathlib4. This file constructs a 1-hypercover of a glued scheme in the big Zariski site and provides tools for constructing global sections of sheaves from local sections using the glue data structure. This is a sophisticated application of descent theory and sheaf cohomology in algebraic geometry.

\section{Background: Sites and Hypercovers}

\subsection{The Big Zariski Site}

In algebraic geometry, the big Zariski site on schemes consists of:
\begin{itemize}
\item Objects: All schemes
\item Morphisms: All scheme morphisms  
\item Covering families: Collections of open immersions that are jointly surjective
\end{itemize}

The Zariski topology is fundamental for studying sheaves on schemes and their cohomological properties.

\subsection{Hypercovers and Descent}

A hypercover is a generalization of a covering that provides a systematic way to study descent and cohomology. A 1-hypercover consists of:
\begin{itemize}
\item A covering of the base scheme
\item Coverings of all pairwise intersections of the covering elements
\item Compatibility conditions ensuring proper descent behavior
\end{itemize}

\section{The One-Hypercover Construction}

\subsection{Main Definition}

\begin{lstlisting}
@[simps]
noncomputable def oneHypercover : Scheme.zariskiTopology.OneHypercover D.glued where
  I₀ := D.J
  X := D.U
  f := D.ι
  I₁ _ _ := PUnit
  Y i₁ i₂ _ := D.V (i₁, i₂)
  p₁ i₁ i₂ _ := D.f i₁ i₂
  p₂ i₁ i₂ _ := D.t i₁ i₂ ≫ D.f i₂ i₁
  w i₁ i₂ _ := by simp only [Category.assoc, Scheme.GlueData.glue_condition]
  mem₀ := ...
  mem₁ := ...
\end{lstlisting}

\textbf{Natural Language:} Given glue data $D$, we construct a 1-hypercover of $D.glued$ as follows:
\begin{itemize}
\item \textbf{Level 0 covering}: The schemes $\{U_i\}_{i \in J}$ with inclusion maps $\iota_i: U_i \to D.glued$
\item \textbf{Level 1 covering}: For each pair $(i,j)$, the single scheme $V_{ij}$ covers the intersection $U_i \times_{D.glued} U_j$
\item \textbf{Projection maps}: $p_1: V_{ij} \to U_i$ and $p_2: V_{ij} \to U_j$ given by $f_{ij}$ and $t_{ij} \circ f_{ji}$
\end{itemize}

\subsection{Level 0 Covering Condition}

\begin{lstlisting}
mem₀ := by
  refine zariskiTopology.superset_covering ?_ (grothendieckTopology_cover D.openCover)
  rw [Sieve.generate_le_iff]
  rintro W _ ⟨i⟩
  exact ⟨_, 𝟙 _, _, ⟨i⟩, by simp; rfl⟩
\end{lstlisting}

\textbf{Natural Language:} The collection $\{U_i \to D.glued\}$ forms a covering in the Zariski topology. This follows from the fact that the $U_i$ form an open cover of $D.glued$, and open covers generate covering sieves in the Zariski topology.

\subsection{Level 1 Covering Condition}

\begin{lstlisting}
mem₁ i₁ i₂ W p₁ p₂ fac := by
  refine zariskiTopology.superset_covering (fun T g _ ↦ ?_) (zariskiTopology.top_mem _)
  have ⟨φ, h₁, h₂⟩ := PullbackCone.IsLimit.lift' (D.vPullbackConeIsLimit i₁ i₂)
    (g ≫ p₁) (g ≫ p₂) (by simpa using g ≫= fac)
  exact ⟨⟨⟩, φ, h₁.symm, h₂.symm⟩
\end{lstlisting}

\textbf{Natural Language:} For any scheme $W$ mapping to both $U_{i_1}$ and $U_{i_2}$ with compatible maps to $D.glued$, this map factors through $V_{i_1 i_2}$. This uses the fact that $V_{i_1 i_2}$ is the pullback (intersection) of $U_{i_1}$ and $U_{i_2}$ over $D.glued$.

\section{Sheaf Value Construction}

\subsection{Global Section Constructor}

\begin{lstlisting}
noncomputable def sheafValGluedMk : F.val.obj (op D.glued) :=
  Multifork.IsLimit.sectionsEquiv (D.oneHypercover.isLimitMultifork F)
    { val := s
      property := fun _ ↦ h _ _ }
\end{lstlisting}

\textbf{Natural Language:} Given:
\begin{itemize}
\item A sheaf $F$ on the big Zariski site
\item Local sections $s_i \in F(U_i)$ for each piece $U_i$
\item Compatibility conditions ensuring $s_i$ and $s_j$ agree on overlaps $V_{ij}$
\end{itemize}
We can construct a global section of $F$ over $D.glued$.

\subsection{Input Data Structure}

\begin{lstlisting}
variable {F : Sheaf Scheme.zariskiTopology (Type v)}
  (s : ∀ (j : D.J), F.val.obj (op (D.U j)))
  (h : ∀ (i j : D.J), F.val.map (D.f i j).op (s i) =
    F.val.map ((D.f j i).op ≫ (D.t i j).op) (s j))
\end{lstlisting}

\textbf{Natural Language:} The input consists of:
\begin{itemize}
\item \textbf{Local sections}: For each index $j$, a section $s_j$ of the sheaf $F$ over $U_j$
\item \textbf{Compatibility condition}: For each pair $(i,j)$, the restrictions of $s_i$ and $s_j$ to $V_{ij}$ must agree under the appropriate restriction maps
\end{itemize}

\subsection{Naturality of Construction}

\begin{lstlisting}
@[simp]
lemma sheafValGluedMk_val (j : D.J) : F.val.map (D.ι j).op (D.sheafValGluedMk s h) = s j
\end{lstlisting}

\textbf{Natural Language:} The constructed global section restricts to the original local section on each piece $U_j$. This confirms that our construction is the correct "gluing" of the local sections.

\section{Theoretical Foundation}

\subsection{Descent Theory Connection}

The construction realizes a fundamental principle of descent theory:
\begin{itemize}
\item \textbf{Local data}: Sections of a sheaf on the pieces of a covering
\item \textbf{Compatibility}: Sections agree on overlaps  
\item \textbf{Global extension}: Unique global section extending the local data
\end{itemize}

This is a special case of the general descent theorem for sheaves on sites.

\subsection{Multifork Limits}

\begin{lstlisting}
Multifork.IsLimit.sectionsEquiv (D.oneHypercover.isLimitMultifork F)
\end{lstlisting}

\textbf{Natural Language:} The construction uses the fact that the hypercover gives rise to a limit diagram (multifork) whose limit computes global sections from compatible local sections. This is the categorical formulation of the sheaf condition.

\section{Applications and Use Cases}

\subsection{Gluing Local Constructions}

This construction is particularly useful for:
\begin{itemize}
\item \textbf{Vector bundles}: Gluing local trivializations
\item \textbf{Line bundles}: Constructing global line bundles from local data
\item \textbf{Morphisms}: Extending locally defined morphisms to global ones
\item \textbf{Sections}: Building global sections from compatible local sections
\end{itemize}

\subsection{Cohomological Applications}

The hypercover provides tools for:
\begin{itemize}
\item Computing sheaf cohomology via Čech complexes
\item Studying descent obstruction and descent data
\item Analyzing the global structure of schemes via local pieces
\item Connecting local and global properties of geometric objects
\end{itemize}

\section{Connection to Classical Theory}

\subsection{Čech Cohomology}

The 1-hypercover construction is closely related to Čech cohomology:
\begin{itemize}
\item The level 0 data corresponds to Čech 0-cochains
\item The compatibility conditions correspond to the Čech differential
\item Global sections correspond to Čech 0-cohomology classes
\end{itemize}

\subsection{Descent for Sheaves}

This formalizes the classical principle that:
\begin{itemize}
\item Sheaves satisfy descent with respect to open covers
\item Local sections can be uniquely extended to global sections if they are compatible
\item The sheaf condition can be checked on covers
\end{itemize}

\section{Technical Implementation Details}

\subsection{Universe Management}

The construction carefully handles universe levels to ensure compatibility across different categorical contexts:
\begin{itemize}
\item Schemes live in universe level $u$
\item Types for sheaf values live in universe level $v$ 
\item The categorical machinery respects these universe constraints
\end{itemize}

\subsection{Noncomputability}

\begin{lstlisting}
noncomputable def sheafValGluedMk : F.val.obj (op D.glued)
\end{lstlisting}

\textbf{Natural Language:} The construction is marked as noncomputable because it relies on:
\begin{itemize}
\item Limit constructions in arbitrary categories
\item Existence proofs that don't provide explicit algorithms
\item Category-theoretic machinery that abstracts away computational content
\end{itemize}

\section{Relationship to Other Gluing}

\subsection{Connection to Basic Gluing}

This construction complements the basic gluing theory:
\begin{itemize}
\item \textbf{Basic gluing}: Constructs schemes from local pieces
\item \textbf{Hypercover gluing}: Provides tools for working with sheaves on glued schemes
\item \textbf{Combined power}: Complete framework for local-to-global constructions
\end{itemize}

\subsection{Site-Theoretic Perspective}

The hypercover approach provides:
\begin{itemize}
\item A systematic way to work with coverings in the Zariski site
\item Tools for computing sheaf cohomology
\item Connections to topos theory and categorical logic
\item A foundation for more advanced descent constructions
\end{itemize}

\section{Conclusion}

The \texttt{GluingOneHypercover.lean} file provides:

\subsection{Theoretical Contributions}
\begin{itemize}
\item Rigorous construction of 1-hypercovers from glue data
\item Proof that glue data provides legitimate coverings in the Zariski site
\item Tools for descent theory and sheaf cohomology
\item Connection between categorical limits and geometric constructions
\end{itemize}

\subsection{Practical Tools}
\begin{itemize}
\item Constructor for global sections from compatible local sections
\item Framework for extending local constructions globally
\item Tools for working with sheaves on glued schemes
\item Foundation for cohomological calculations
\end{itemize}

\subsection{Mathematical Significance}
\begin{itemize}
\item Formalizes fundamental principles of descent theory
\item Provides computational tools for sheaf theory
\item Establishes the connection between gluing and site theory
\item Enables advanced constructions in algebraic geometry
\end{itemize}

This work represents a sophisticated application of category theory and topos theory to algebraic geometry, providing both theoretical foundations and practical tools for working with sheaves and descent in the context of scheme theory. The construction is essential for understanding how local algebraic and geometric data can be systematically assembled into global objects.

\end{document>