\documentclass{article}
\usepackage{amsmath, amssymb, amsthm}
\usepackage{hyperref}
\usepackage{listings}
\usepackage{xcolor}

\theoremstyle{definition}
\newtheorem{definition}{Definition}
\newtheorem{theorem}{Theorem}
\newtheorem{lemma}{Lemma}
\newtheorem{example}{Example}

\title{Analysis Module: AsymptoticEquivalent}
\author{Mathlib4 Documentation}
\date{\today}

\begin{document}
\maketitle

\section{Module Overview}
\subsection{\1}

In this file, we define the relation \texttt{\1}, which means that \texttt{\1} is little o of
\texttt{\1} along the filter \texttt{\1}.

Unlike \texttt{\1} relations, this one requires \texttt{\1} and \texttt{\1} to have the same codomain \texttt{\1}.
While the definition only requires \texttt{\1} to be a \texttt{\1}, most interesting properties
require it to be a \texttt{\1}.

#\subsection{\1}

We introduce the notation \texttt{\1}, which you can use by opening the
\texttt{\1} locale.

#\subsection{\1}

If \texttt{\1} is a \texttt{\1} :

- \texttt{\1} is an equivalence relation
- Equivalent statements for \texttt{\1} :
  - If \texttt{\1}, this is true iff \texttt{\1} (see \texttt{\1})
  - For \texttt{\1}, this is true iff \texttt{\1} (see \texttt{\1})

If \texttt{\1} is a \texttt{\1} :

- Alternative characterization of the relation (see \texttt{\1}) :

  \texttt{\1}

- Provided some non-vanishing hypothesis, this can be seen as \texttt{\1}
  (see \texttt{\1})
- For any constant \texttt{\1}, \texttt{\1} implies \texttt{\1}
  (see \texttt{\1})
- \texttt{\1} and \texttt{\1} are compatible with \texttt{\1} (see \texttt{\1} and \texttt{\1})

If \texttt{\1} is a \texttt{\1} :

- If \texttt{\1}, we have \texttt{\1}
  (see \texttt{\1})

#\subsection{\1}

Note that \texttt{\1} takes the parameters \texttt{\1} in that order.
This is to enable \texttt{\1} \support, as \texttt{\1} requires that the last two explicit arguments are \texttt{\1}.

\section{Key Definitions}
\begin{definition}[IsEquivalent]
A def defining \texttt{IsEquivalent}
\end{definition}

\begin{definition}[IsEquivalent]
A theorem defining \texttt{IsEquivalent}
\end{definition}

\begin{definition}[IsEquivalent]
A theorem defining \texttt{IsEquivalent}
\end{definition}

\begin{definition}[IsEquivalent]
A theorem defining \texttt{IsEquivalent}
\end{definition}

\begin{definition}[IsEquivalent]
A theorem defining \texttt{IsEquivalent}
\end{definition}

\begin{definition}[IsEquivalent]
A theorem defining \texttt{IsEquivalent}
\end{definition}

\begin{definition}[IsEquivalent]
A theorem defining \texttt{IsEquivalent}
\end{definition}

\begin{definition}[IsEquivalent]
A theorem defining \texttt{IsEquivalent}
\end{definition}

\begin{definition}[IsEquivalent]
A theorem defining \texttt{IsEquivalent}
\end{definition}

\begin{definition}[IsEquivalent]
A theorem defining \texttt{IsEquivalent}
\end{definition}


\end{document}
