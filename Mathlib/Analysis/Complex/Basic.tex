\documentclass{article}
\usepackage{amsmath, amssymb, amsthm}
\usepackage{hyperref}
\usepackage{listings}
\usepackage{xcolor}

\theoremstyle{definition}
\newtheorem{definition}{Definition}
\newtheorem{theorem}{Theorem}
\newtheorem{lemma}{Lemma}
\newtheorem{example}{Example}

\title{Analysis Module: Basic}
\author{Mathlib4 Documentation}
\date{\today}

\begin{document}
\maketitle

\section{Module Overview}
\subsection{\1}

This file gathers basic facts of analytic nature on the complex numbers.

#\subsection{\1}

This file registers \texttt{\1} as a normed field, expresses basic properties of the norm, and gives tools
on the real vector space structure of \texttt{\1}. Notably, it defines the following functions in the
namespace \texttt{\1}.

|Name              |Type         |Description                                             |
|------------------|-------------|--------------------------------------------------------|
|\texttt{\1}|\mathbb{C} ≃L[\mathbb{R}] \mathbb{R} \times \mathbb{R}|The natural \texttt{\1} from \texttt{\1} to \texttt{\1} |
|\texttt{\1}           |\mathbb{C} \toL[\mathbb{R}] \mathbb{R}    |Real part function as a \texttt{\1}           |
|\texttt{\1}           |\mathbb{C} \toL[\mathbb{R}] \mathbb{R}    |Imaginary part function as a \texttt{\1}      |
|\texttt{\1}       |\mathbb{R} \toL[\mathbb{R}] \mathbb{C}    |Embedding of the reals as a \texttt{\1}       |
|\texttt{\1}        |\mathbb{R} \toₗᵢ[\mathbb{R}] \mathbb{C}   |Embedding of the reals as a \texttt{\1}            |
|\texttt{\1}         |\mathbb{C} ≃L[\mathbb{R}] \mathbb{C}    |Complex conjugation as a \texttt{\1}        |
|\texttt{\1}         |\mathbb{C} ≃ₗᵢ[\mathbb{R}] \mathbb{C}   |Complex conjugation as a \texttt{\1}          |

We also register the fact that \texttt{\1} is an \texttt{\1} field.

\section{Key Definitions}
\begin{definition}[continuous_normSq]
A theorem defining \texttt{continuous_normSq}
\end{definition}

\begin{definition}[nnnorm_eq_one_of_pow_eq_one]
A theorem defining \texttt{nnnorm_eq_one_of_pow_eq_one}
\end{definition}

\begin{definition}[norm_eq_one_of_pow_eq_one]
A theorem defining \texttt{norm_eq_one_of_pow_eq_one}
\end{definition}

\begin{definition}[le_of_eq_sum_of_eq_sum_norm]
A lemma defining \texttt{le_of_eq_sum_of_eq_sum_norm}
\end{definition}

\begin{definition}[equivRealProd_apply_le]
A theorem defining \texttt{equivRealProd_apply_le}
\end{definition}

\begin{definition}[equivRealProd_apply_le]
A theorem defining \texttt{equivRealProd_apply_le}
\end{definition}

\begin{definition}[lipschitz_equivRealProd]
A theorem defining \texttt{lipschitz_equivRealProd}
\end{definition}

\begin{definition}[antilipschitz_equivRealProd]
A theorem defining \texttt{antilipschitz_equivRealProd}
\end{definition}

\begin{definition}[isUniformEmbedding_equivRealProd]
A theorem defining \texttt{isUniformEmbedding_equivRealProd}
\end{definition}

\begin{definition}[equivRealProdCLM]
A def defining \texttt{equivRealProdCLM}
\end{definition}

\section{Main Theorems}
\begin{theorem}[tendsto_normSq_cocompact_atTop]
The `normSq` function on `\mathbb{C}` is proper.
\end{theorem}

\begin{theorem}[ringHom_eq_ofReal_of_continuous]
The only continuous ring homomorphism from `\mathbb{R}` to `\mathbb{C}` is the identity.
\end{theorem}

\begin{theorem}[ball_one_subset_slitPlane]
The slit plane includes the open unit ball of radius `1` around `1`.
\end{theorem}

\begin{theorem}[mem_slitPlane_of_norm_lt_one]
The slit plane includes the open unit ball of radius `1` around `1`.
\end{theorem}


\end{document}