\documentclass{article}
\usepackage{amsmath, amssymb, amsthm}
\usepackage{hyperref}
\usepackage{listings}
\usepackage{xcolor}

\theoremstyle{definition}
\newtheorem{definition}{Definition}
\newtheorem{theorem}{Theorem}
\newtheorem{lemma}{Lemma}
\newtheorem{example}{Example}

\title{Analysis Module: FourierTransform}
\author{Mathlib4 Documentation}
\date{\today}

\begin{document}
\maketitle

\section{Module Overview}
\subsection{\1}

We set up the Fourier transform for complex-valued functions on finite-dimensional spaces.

#\subsection{\1}

In namespace \texttt{\1}, we define the Fourier integral in the following context:
* \texttt{\1} is a commutative ring.
* \texttt{\1} and \texttt{\1} are \texttt{\1}-modules.
* \texttt{\1} is a unitary additive character of \texttt{\1}, i.e. an \texttt{\1}.
* \texttt{\1} is a measure on \texttt{\1}.
* \texttt{\1} is a \texttt{\1}-bilinear form \texttt{\1}.
* \texttt{\1} is a complete normed \texttt{\1}-vector space.

With these definitions, we define \texttt{\1} to be the map from functions \texttt{\1} to
functions \texttt{\1} that sends \texttt{\1} to

\texttt{\1},

This includes the cases \texttt{\1} is the dual of \texttt{\1} and \texttt{\1} is the canonical pairing, or \texttt{\1} and \texttt{\1}
is a bilinear form (e.g. an inner product).

In namespace \texttt{\1}, we consider the more familiar special case when \texttt{\1} and \texttt{\1} is the
multiplication map (but still allowing \texttt{\1} to be an arbitrary ring equipped with a measure).

The most familiar case of all is when \texttt{\1}, \texttt{\1} is multiplication, \texttt{\1} is volume, and
\texttt{\1} is \texttt{\1}, i.e. the character \texttt{\1} (for which we
introduced the notation \texttt{\1} in the scope \texttt{\1}).

Another familiar case (which generalizes the previous one) is when \texttt{\1} is an inner product space
over \texttt{\1} and \texttt{\1} is the scalar product. We introduce two notations \texttt{\1} for the Fourier transform in
this case and \texttt{\1} for the inverse Fourier transform. These notations make
in particular sense for \texttt{\1}.

#\subsection{\1}

At present the only nontrivial lemma we prove is \texttt{\1}, stating that the
Fourier transform of an integrable function is continuous (under mild assumptions).

\section{Key Definitions}
\begin{definition}[fourierIntegral]
A def defining \texttt{fourierIntegral}
\end{definition}

\begin{definition}[fourierIntegral_const_smul]
A theorem defining \texttt{fourierIntegral_const_smul}
\end{definition}

\begin{definition}[norm_fourierIntegral_le_integral_norm]
A theorem defining \texttt{norm_fourierIntegral_le_integral_norm}
\end{definition}

\begin{definition}[fourierIntegral_comp_add_right]
A theorem defining \texttt{fourierIntegral_comp_add_right}
\end{definition}

\begin{definition}[fourierIntegral_convergent_iff]
A theorem defining \texttt{fourierIntegral_convergent_iff}
\end{definition}

\begin{definition}[fourierIntegral_add]
A theorem defining \texttt{fourierIntegral_add}
\end{definition}

\begin{definition}[fourierIntegral_continuous]
A theorem defining \texttt{fourierIntegral_continuous}
\end{definition}

\begin{definition}[integral_bilin_fourierIntegral_eq_flip]
A theorem defining \texttt{integral_bilin_fourierIntegral_eq_flip}
\end{definition}

\begin{definition}[integral_fourierIntegral_smul_eq_flip]
A theorem defining \texttt{integral_fourierIntegral_smul_eq_flip}
\end{definition}

\begin{definition}[fourierIntegral_probChar]
A lemma defining \texttt{fourierIntegral_probChar}
\end{definition}

\section{Main Theorems}
\begin{theorem}[norm_fourierIntegral_le_integral_norm]
The uniform norm of the Fourier integral of `f` is bounded by the `L¹` norm of `f`.
\end{theorem}

\begin{theorem}[fourierIntegral_comp_add_right]
The Fourier integral converts right-translation into scalar multiplication by a phase factor.
\end{theorem}

\begin{theorem}[fourierIntegral_convergent_iff]
For any `w`, the Fourier integral is convergent iff `f` is integrable.
\end{theorem}

\begin{theorem}[fourierIntegral_continuous]
The Fourier integral of an `L^1` function is a continuous function.
\end{theorem}

\begin{theorem}[integral_fourierIntegral_smul_eq_flip]
The Fourier transform satisfies `\int \mathcal{F} f * g = \int f * \mathcal{F} g`, i.e., it is self-adjoint.
\end{theorem}

\begin{theorem}[norm_fourierIntegral_le_integral_norm]
The uniform norm of the Fourier transform of `f` is bounded by the `L¹` norm of `f`.
\end{theorem}


\end{document}