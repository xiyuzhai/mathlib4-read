\documentclass{article}
\usepackage{amsmath, amssymb, amsthm}
\usepackage{hyperref}
\usepackage{listings}
\usepackage{xcolor}

\theoremstyle{definition}
\newtheorem{definition}{Definition}
\newtheorem{theorem}{Theorem}
\newtheorem{lemma}{Lemma}
\newtheorem{example}{Example}

\title{Analysis Module: Basic}
\author{Mathlib4 Documentation}
\date{\today}

\begin{document}
\maketitle

\section{Module Overview}
\subsection{Gradient}

\subsubsection{Main Definitions}

Let \\texttt{f} be a function from a Hilbert Space \\texttt{F} to \\texttt{\mathbb{K}} (\\texttt{\mathbb{K}} is \\texttt{\mathbb{R}} or \\texttt{\mathbb{C}}), \\texttt{x} be a point in \\texttt{F}
and \\texttt{f'} be a vector in F. Then

  \\texttt{HasGradientWithinAt f f' s x}

says that \\texttt{f} has a gradient \\texttt{f'} at \\texttt{x}, where the domain of interest
is restricted to \\texttt{s}. We also have

  \\texttt{HasGradientAt f f' x := HasGradientWithinAt f f' x univ}

\subsubsection{Main results}

This file contains the following parts of gradient.
* the definition of gradient.
* the theorems translating between \\texttt{HasGradientAtFilter} and \\texttt{HasFDerivAtFilter},
  \\texttt{HasGradientWithinAt} and \\texttt{HasFDerivWithinAt}, \\texttt{HasGradientAt} and \\texttt{HasFDerivAt},
  \\texttt{Gradient} and \\texttt{fderiv}.
* theorems the Uniqueness of Gradient.
* the theorems translating between  \\texttt{HasGradientAtFilter} and \\texttt{HasDerivAtFilter},
  \\texttt{HasGradientAt} and \\texttt{HasDerivAt}, \\texttt{Gradient} and \\texttt{deriv} when \\texttt{F = \mathbb{K}}.
* the theorems about the congruence of the gradient.
* the theorems about the gradient of constant function.
* the theorems about the continuity of a function admitting a gradient.

\section{Key Definitions}
\begin{definition}[HasGradientAtFilter]
A def defining \texttt{HasGradientAtFilter}
\end{definition}

\begin{definition}[HasGradientWithinAt]
A def defining \texttt{HasGradientWithinAt}
\end{definition}

\begin{definition}[HasGradientAt]
A def defining \texttt{HasGradientAt}
\end{definition}

\begin{definition}[gradientWithin]
A def defining \texttt{gradientWithin}
\end{definition}

\begin{definition}[gradient]
A def defining \texttt{gradient}
\end{definition}

\begin{definition}[hasGradientWithinAt_iff_hasFDerivWithinAt]
A theorem defining \texttt{hasGradientWithinAt_iff_hasFDerivWithinAt}
\end{definition}

\begin{definition}[hasFDerivWithinAt_iff_hasGradientWithinAt]
A theorem defining \texttt{hasFDerivWithinAt_iff_hasGradientWithinAt}
\end{definition}

\begin{definition}[hasGradientAt_iff_hasFDerivAt]
A theorem defining \texttt{hasGradientAt_iff_hasFDerivAt}
\end{definition}

\begin{definition}[hasFDerivAt_iff_hasGradientAt]
A theorem defining \texttt{hasFDerivAt_iff_hasGradientAt}
\end{definition}

\begin{definition}[gradient_eq_zero_of_not_differentiableAt]
A theorem defining \texttt{gradient_eq_zero_of_not_differentiableAt}
\end{definition}

\section{Derivatives}
This module deals with derivatives of functions $f : \mathbb{K} \to F$ where:
\begin{itemize}
\item $\mathbb{K}$ is a normed field
\item $F$ is a normed space over $\mathbb{K}$
\end{itemize}

\subsection{Notations}
\begin{itemize}
\item $\text{deriv } f$ denotes the derivative of $f$
\item $\text{HasDerivAt } f \, f' \, x$ means $f'(x) = f'$
\item $\text{derivWithin } f \, s \, x$ is the derivative within set $s$
\end{itemize}


\end{document}