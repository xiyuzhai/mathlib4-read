\documentclass{article}
\usepackage{amsmath, amssymb, amsthm}
\usepackage{hyperref}
\usepackage{listings}
\usepackage{xcolor}

\theoremstyle{definition}
\newtheorem{definition}{Definition}
\newtheorem{theorem}{Theorem}
\newtheorem{lemma}{Lemma}
\newtheorem{example}{Example}

\title{Analysis Module: Basic}
\author{Mathlib4 Documentation}
\date{\today}

\begin{document}
\maketitle

\section{Module Overview}
\subsection{Local convexity}

This file defines absorbent and balanced sets.

An absorbent set is one that "surrounds" the origin. The idea is made precise by requiring that any
point belongs to all large enough scalings of the set. This is the vector world analog of a
topological neighborhood of the origin.

A balanced set is one that is everywhere around the origin. This means that \\texttt{a • s \subseteq s} for all \\texttt{a}
of norm less than \\texttt{1}.

\subsubsection{Main declarations}

For a module over a normed ring:
* \\texttt{Absorbs}: A set \\texttt{s} absorbs a set \\texttt{t} if all large scalings of \\texttt{s} contain \\texttt{t}.
* \\texttt{Absorbent}: A set \\texttt{s} is absorbent if every point eventually belongs to all large scalings of
  \\texttt{s}.
* \\texttt{Balanced}: A set \\texttt{s} is balanced if \\texttt{a • s \subseteq s} for all \\texttt{a} of norm less than \\texttt{1}.

\subsubsection{References}

* [H. H. Schaefer, *Topological Vector Spaces*][schaefer1966]

\subsubsection{Tags}

absorbent, balanced, locally convex, LCTVS

\section{Key Definitions}
\begin{definition}[Balanced]
A def defining \texttt{Balanced}
\end{definition}

\begin{definition}[absorbs_iff_norm]
A lemma defining \texttt{absorbs_iff_norm}
\end{definition}

\begin{definition}[Absorbs]
A lemma defining \texttt{Absorbs}
\end{definition}

\begin{definition}[balanced_iff_smul_mem]
A theorem defining \texttt{balanced_iff_smul_mem}
\end{definition}

\begin{definition}[balanced_iff_closedBall_smul]
A theorem defining \texttt{balanced_iff_closedBall_smul}
\end{definition}

\begin{definition}[balanced_empty]
A theorem defining \texttt{balanced_empty}
\end{definition}

\begin{definition}[balanced_univ]
A theorem defining \texttt{balanced_univ}
\end{definition}

\begin{definition}[Balanced]
A theorem defining \texttt{Balanced}
\end{definition}

\begin{definition}[Balanced]
A theorem defining \texttt{Balanced}
\end{definition}

\begin{definition}[balanced_iUnion]
A theorem defining \texttt{balanced_iUnion}
\end{definition}

\section{Main Theorems}
\begin{theorem}[Balanced]
Scalar multiplication (by possibly different types) of a balanced set is monotone.
\end{theorem}

\begin{theorem}[Balanced]
A balanced set absorbs itself.
\end{theorem}

\begin{theorem}[absorbent_nhds_zero]
Every neighbourhood of the origin is absorbent.
\end{theorem}

\begin{theorem}[Balanced]
The union of `{0}` with the interior of a balanced set is balanced.
\end{theorem}


\end{document}