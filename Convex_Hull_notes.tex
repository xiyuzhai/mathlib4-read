\documentclass{article}
\usepackage{amsmath, amssymb, amsthm}
\usepackage{hyperref}
\usepackage{listings}
\usepackage{xcolor}

\theoremstyle{definition}
\newtheorem{definition}{Definition}
\newtheorem{theorem}{Theorem}
\newtheorem{lemma}{Lemma}
\newtheorem{example}{Example}

\title{Analysis Module: Hull}
\author{Mathlib4 Documentation}
\date{\today}

\begin{document}
\maketitle

\section{Module Overview}
\subsection{Convex hull}

This file defines the convex hull of a set \\texttt{s} in a module. \\texttt{convexHull \mathbb{K} s} is the smallest convex
set containing \\texttt{s}. In order theory speak, this is a closure operator.

\subsubsection{Implementation notes}

\\texttt{convexHull} is defined as a closure operator. This gives access to the \\texttt{ClosureOperator} API
while the impact on writing code is minimal as \\texttt{convexHull \mathbb{K} s} is automatically elaborated as
\\texttt{(convexHull \mathbb{K}) s}.

\section{Key Definitions}
\begin{definition}[convexHull]
A def defining \texttt{convexHull}
\end{definition}

\begin{definition}[subset_convexHull]
A theorem defining \texttt{subset_convexHull}
\end{definition}

\begin{definition}[convex_convexHull]
A theorem defining \texttt{convex_convexHull}
\end{definition}

\begin{definition}[convexHull_eq_iInter]
A theorem defining \texttt{convexHull_eq_iInter}
\end{definition}

\begin{definition}[mem_convexHull_iff]
A theorem defining \texttt{mem_convexHull_iff}
\end{definition}

\begin{definition}[convexHull_min]
A theorem defining \texttt{convexHull_min}
\end{definition}

\begin{definition}[Convex]
A theorem defining \texttt{Convex}
\end{definition}

\begin{definition}[convexHull_mono]
A theorem defining \texttt{convexHull_mono}
\end{definition}

\begin{definition}[convexHull_eq_self]
A lemma defining \texttt{convexHull_eq_self}
\end{definition}

\begin{definition}[convexHull_univ]
A theorem defining \texttt{convexHull_univ}
\end{definition}


\end{document}