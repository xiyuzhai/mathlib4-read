\documentclass{article}
\usepackage{amsmath, amssymb, amsthm}
\usepackage{hyperref}
\usepackage{listings}
\usepackage{xcolor}

\theoremstyle{definition}
\newtheorem{definition}{Definition}
\newtheorem{theorem}{Theorem}
\newtheorem{lemma}{Lemma}
\newtheorem{example}{Example}

\title{Analysis Module: Basic}
\author{Mathlib4 Documentation}
\date{\today}

\begin{document}
\maketitle

\section{Module Overview}
\subsection{\1}

This file proves many basic properties of inner product spaces (real or complex).

#\subsection{\1}

- \texttt{\1}: the Cauchy-Schwartz inequality (one of many variants).
- \texttt{\1}: the equality criteion in the Cauchy-Schwartz inequality (also in many
  variants).
- \texttt{\1}: the polarization identity.

#\subsection{\1}

inner product space, Hilbert space, norm

\section{Key Definitions}
\begin{definition}[inner_conj_symm]
A theorem defining \texttt{inner_conj_symm}
\end{definition}

\begin{definition}[real_inner_comm]
A theorem defining \texttt{real_inner_comm}
\end{definition}

\begin{definition}[inner_eq_zero_symm]
A theorem defining \texttt{inner_eq_zero_symm}
\end{definition}

\begin{definition}[inner_self_im]
A theorem defining \texttt{inner_self_im}
\end{definition}

\begin{definition}[inner_add_left]
A theorem defining \texttt{inner_add_left}
\end{definition}

\begin{definition}[inner_add_right]
A theorem defining \texttt{inner_add_right}
\end{definition}

\begin{definition}[inner_re_symm]
A theorem defining \texttt{inner_re_symm}
\end{definition}

\begin{definition}[inner_im_symm]
A theorem defining \texttt{inner_im_symm}
\end{definition}

\begin{definition}[inner_smul_left_eq_star_smul]
A lemma defining \texttt{inner_smul_left_eq_star_smul}
\end{definition}

\begin{definition}[inner_smul_left_eq_smul]
A lemma defining \texttt{inner_smul_left_eq_smul}
\end{definition}

\section{Main Theorems}
\begin{theorem}[inner_smul_left_eq_star_smul]
See `inner_smul_left` for the common special when `\mathbb{K} = 𝕝`.
\end{theorem}

\begin{theorem}[inner_smul_right_eq_smul]
See `inner_smul_right` for the common special when `\mathbb{K} = 𝕝`.
\end{theorem}

\begin{theorem}[inner_smul_left]
See `inner_smul_left_eq_star_smul` for the case of a general algebra action.
\end{theorem}

\begin{theorem}[inner_smul_right]
See `inner_smul_right_eq_smul` for the case of a general algebra action.
\end{theorem}

\begin{theorem}[sum_inner]
An inner product with a sum on the left.
\end{theorem}

\begin{theorem}[inner_sum]
An inner product with a sum on the right.
\end{theorem}

\begin{theorem}[inner_add_add_self]
Expand `⟪x + y, x + y⟫`
\end{theorem}

\begin{theorem}[real_inner_add_add_self]
Expand `⟪x + y, x + y⟫_\mathbb{R}`
\end{theorem}

\begin{theorem}[real_inner_sub_sub_self]
Expand `⟪x - y, x - y⟫_\mathbb{R}`
\end{theorem}

\begin{theorem}[parallelogram_law]
Parallelogram law
\end{theorem}


\end{document}