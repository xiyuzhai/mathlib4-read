\documentclass{article}
\usepackage{amsmath, amssymb, amsthm}
\usepackage{hyperref}
\usepackage{listings}
\usepackage{xcolor}

\theoremstyle{definition}
\newtheorem{definition}{Definition}
\newtheorem{theorem}{Theorem}
\newtheorem{lemma}{Lemma}
\newtheorem{example}{Example}

\title{Analysis Module: Separation}
\author{Mathlib4 Documentation}
\date{\today}

\begin{document}
\maketitle

\section{Module Overview}
\subsection{Separation Hahn-Banach theorem}

In this file we prove the geometric Hahn-Banach theorem. For any two disjoint convex sets, there
exists a continuous linear functional separating them, geometrically meaning that we can intercalate
a plane between them.

We provide many variations to stricten the result under more assumptions on the convex sets:
* \\texttt{geometric\\_hahn\\_banach\\_open}: One set is open. Weak separation.
* \\texttt{geometric\\_hahn\\_banach\\_open\\_point}, \\texttt{geometric\\_hahn\\_banach\\_point\\_open}: One set is open, the
  other is a singleton. Weak separation.
* \\texttt{geometric\\_hahn\\_banach\\_open\\_open}: Both sets are open. Semistrict separation.
* \\texttt{geometric\\_hahn\\_banach\\_compact\\_closed}, \\texttt{geometric\\_hahn\\_banach\\_closed\\_compact}: One set is closed,
  the other one is compact. Strict separation.
* \\texttt{geometric\\_hahn\\_banach\\_point\\_closed}, \\texttt{geometric\\_hahn\\_banach\\_closed\\_point}: One set is closed, the
  other one is a singleton. Strict separation.
* \\texttt{geometric\\_hahn\\_banach\\_point\\_point}: Both sets are singletons. Strict separation.

\subsubsection{TODO}

* Eidelheit's theorem
* \\texttt{Convex \mathbb{R} s \to interior (closure s) \subseteq s}

\section{Key Definitions}
\begin{definition}[separate_convex_open_set]
A theorem defining \texttt{separate_convex_open_set}
\end{definition}

\begin{definition}[geometric_hahn_banach_open]
A theorem defining \texttt{geometric_hahn_banach_open}
\end{definition}

\begin{definition}[geometric_hahn_banach_open_point]
A theorem defining \texttt{geometric_hahn_banach_open_point}
\end{definition}

\begin{definition}[geometric_hahn_banach_point_open]
A theorem defining \texttt{geometric_hahn_banach_point_open}
\end{definition}

\begin{definition}[geometric_hahn_banach_open_open]
A theorem defining \texttt{geometric_hahn_banach_open_open}
\end{definition}

\begin{definition}[geometric_hahn_banach_compact_closed]
A theorem defining \texttt{geometric_hahn_banach_compact_closed}
\end{definition}

\begin{definition}[geometric_hahn_banach_closed_compact]
A theorem defining \texttt{geometric_hahn_banach_closed_compact}
\end{definition}

\begin{definition}[geometric_hahn_banach_point_closed]
A theorem defining \texttt{geometric_hahn_banach_point_closed}
\end{definition}

\begin{definition}[geometric_hahn_banach_closed_point]
A theorem defining \texttt{geometric_hahn_banach_closed_point}
\end{definition}

\begin{definition}[geometric_hahn_banach_point_point]
A theorem defining \texttt{geometric_hahn_banach_point_point}
\end{definition}

\section{Main Theorems}
\begin{theorem}[geometric_hahn_banach_point_point]
See also `NormedSpace.eq_iff_forall_dual_eq`.
\end{theorem}

\begin{theorem}[iInter_halfSpaces_eq]
A closed convex set is the intersection of the half-spaces containing it.
\end{theorem}


\end{document}